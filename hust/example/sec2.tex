% 第二节

\section{岳阳楼记}\label{sec:test}
\subsection{第一部分}
\subsubsection{第一段}
庆历四年春,滕子京谪守巴陵郡。越明年,政通人和,百废具兴,乃重修岳阳楼,增其旧制,刻唐贤今人诗赋于其上,属予作文以记之。

\subsubsection{第二段}
予观夫巴陵胜状,在洞庭一湖。衔远山,吞长江,浩浩汤汤,横无际涯,朝晖夕阴,气象万千,此则岳阳楼之大观也,前人之述备矣。然则北通巫峡,南极潇湘,迁客骚人,多会于此,览物之情,得无异乎?

\subsection{第二部分}
\subsubsection{第三段}
若夫淫雨霏霏,连月不开,阴风怒号,浊浪排空,日星隐曜,山岳潜形,商旅不行,樯倾楫摧,薄暮冥冥,虎啸猿啼。登斯楼也,则有去国怀乡,忧谗畏讥,满目萧然,感极而悲者矣。

\begin{equation}
    \vec{a}=a_t\vec{\tau}+a_n\vec{n}=\frac{\mathrm{d}v}{\mathrm{d}t}\vec{\tau}+\frac{v^2}{\rho}\vec{n}
\end{equation}

\subsubsection{第四段}
至若春和景明,波澜不惊,上下天光,一碧万顷,沙鸥翔集,锦鳞游泳,岸芷汀兰,郁郁青青。而或长烟一空,皓月千里,浮光跃金,静影沉璧,渔歌互答,此乐何极!登斯楼也,则有心旷神怡,宠辱偕忘,把酒临风,其喜洋洋者矣。

\subsubsection{第五段}
嗟夫!予尝求古仁人之心,或异二者之为,何哉?不以物喜,不以己悲,居庙堂之高则忧其民,处江湖之远则忧其君。是进亦忧,退亦忧。然则何时而乐耶?其必曰“先天下之忧而忧,后天下之乐而乐”乎!噫!微斯人,吾谁与归?

时六年九月十五日。
